% date: 2024-09-21
% Author: Bai Junhao
% Email: bjh2001@connect.hku.hk
% The University of Hong Kong

\section{Importance of the Phonetic Alphabet in Voice Communication}

\subsection{Question}
Explain, by way of illustrating a number of sample spectrograms, why it is necessary to use phonetic alphabet (i..e A- Alpha, B - Beta, C - Charlie, …) to ensure correct exchange of letters over voice messages by radio or telephone.

\subsection{Answer}
The use of the phonetic alphabet (e.g., A for Alpha, B for Bravo, C for Charlie) is essential for ensuring the correct exchange of letters in voice communication, especially over radio or telephone, where audio quality may be compromised. This system provides clarity in distinguishing between letters that sound similar or may be misheard in noisy environments. By examining spectrograms, we can visually understand why this is necessary.

\subsubsection*{Similar Sounding Letters}
Certain letters in the English alphabet have phonemes that are acoustically similar, particularly under poor communication conditions. For instance, letters like "B" and "D" or "M" and "N" have similar stop consonants, and the difference between them may be subtle, especially in environments with background noise or signal interference. In a spectrogram, these letters may show similar patterns, with short bursts of energy followed by a vowel sound, making them hard to distinguish.

When the phonetic alphabet is used, however, these subtle differences become clearer. "B" is replaced by "Bravo," and "D" is replaced by "Delta," which not only introduces longer, distinct vowel sounds but also gives additional context. The spectrograms of "Bravo" and "Delta" show more distinguishable features with clear energy distribution over time, reducing confusion between the letters.

\subsubsection*{Dealing with Noisy Environments}
Radio and telephone communications often suffer from interference, distortion, or bandwidth limitations that can affect the clarity of speech. High-frequency noise or signal dropouts may mask critical parts of the speech signal, particularly affecting the stop consonants or fricatives in single-letter pronunciations like "F" and "S."

In spectrograms of real-world noisy communications, it's common to see sections of the signal interrupted by noise, particularly in the higher frequency ranges where consonants are more prominent. This can lead to misinterpretation of letters. Using a word from the phonetic alphabet like "Foxtrot" instead of "F" and "Sierra" instead of "S" provides more distinct visual and auditory cues, as these words have longer durations, clearer vowel transitions, and lower susceptibility to noise.

\subsubsection*{Clarity of Pronunciation}
Many letters in isolation are pronounced with very short sounds, which can lead to misinterpretation if the transmission quality is low. For instance, the letters "P," "T," and "K" are all short plosive sounds that may look similar on a spectrogram, as they have a rapid energy release followed by a brief period of silence. These letters, when transmitted over a low-quality connection, can be easily confused.

The phonetic alphabet elongates the transmission by incorporating longer, multi-syllabic words. The spectrograms of "Papa," "Tango," and "Kilo" show much more distinctive patterns, with clear vowel sounds separating the consonants. This helps prevent confusion by providing additional auditory information that is less likely to be lost in transmission.

\subsubsection*{Conclusion}
In summary, the use of the phonetic alphabet is crucial for ensuring the accurate exchange of letters over voice messages in challenging conditions. Spectrogram analysis highlights why this system is effective: it reduces ambiguity by replacing short, similar-sounding letters with longer, more distinct words that are easier to distinguish, even in noisy or degraded communication environments. The distinct visual representation of phonetic alphabet words on a spectrogram further demonstrates their clarity and robustness, making them indispensable for reliable voice communication.

\newpage